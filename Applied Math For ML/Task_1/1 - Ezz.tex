\documentclass{article}
\usepackage{amsmath}

\begin{document}

\section*{First part}
\[
A =
\begin{bmatrix}
-1 & 23 & 10 \\
0 & -2 & -11
\end{bmatrix},
\quad
B =
\begin{bmatrix}
-6 & 2 & 10 \\
-3 & -3 & 4 \\
-5 & -11 & 9 \\
1 & -1 & 9
\end{bmatrix},
\quad
C =
\begin{bmatrix}
-3 & 2 & 9 & -5 & 7
\end{bmatrix}
\]

\[
D =
\begin{bmatrix}
-2 & 6 \\
-5 & 2
\end{bmatrix},
\quad
E =
\begin{bmatrix}
3
\end{bmatrix},
\quad
F =
\begin{bmatrix}
3 \\
5 \\
-11 \\
7
\end{bmatrix},
\quad
G =
\begin{bmatrix}
-6 & -4 & 23 \\
-4 & -3 & 4 \\
23 & 4 & 1
\end{bmatrix}
\]


\begin{enumerate}
    \item[(a)] Dimensions of the matrices:
    \[
    \begin{aligned}
    &A: 2 \times 3, \quad B: 4 \times 3, \quad C: 1 \times 5, \\
    &D: 2 \times 2, \quad E: 1 \times 1, \quad F: 4 \times 1, \quad G: 3 \times 3.
    \end{aligned}
    \]

    \item[(b)] Square matrices: \(D, E, G\) 

    \item[(c)] Symmetric matrices: \(D, G\)

    \item[(d)] The matrix with the entry at row 3 and column 2 equal to -11 is \( B \).

    \item[(e)] The matrix with the entry at row 1 and column 3 equal to 10 is \( A \).

    \item[(f)] Column matrices: \( F \) 

    \item[(g)] Row matrices: \( C \) 

    \item[(h)] Transposes:
    \[
    A^T =
    \begin{bmatrix}
    -1 & 0 \\
    23 & -2 \\
    10 & -11
    \end{bmatrix}
    \]
    
    \[
    C^T =
    \begin{bmatrix}
    -3 \\
    2 \\
    9 \\
    -5 \\
    7
    \end{bmatrix}
    \]

    \[
    E^T =
    \begin{bmatrix}
    3
    \end{bmatrix}
    \]

    \[
    G^T =
    \begin{bmatrix}
    -6 & -4 & 23 \\
    -4 & -3 & 4 \\
    23 & 4 & 1
    \end{bmatrix} = G
    \]
\end{enumerate}


\section*{Second part}
\[
A =
\begin{bmatrix}
-1 & 1 & -2 \\
0 & -2 & 1
\end{bmatrix},
\quad
B =
\begin{bmatrix}
-1 & 2 & 0 \\
0 & -3 & 4 \\
-1 & -2 & 3
\end{bmatrix},
\quad
C =
\begin{bmatrix}
-3 & 2 & 9 & -5 & 7
\end{bmatrix}
\]

\[
D =
\begin{bmatrix}
-2 & 6 \\
-5 & 2
\end{bmatrix}
\]


\begin{enumerate}
    \item[(a)] \textbf{AB}: Possible if the number of columns of \(A\) matches the number of rows of \(B\). \(A\) is \(2 \times 3\) and \(B\) is \(3 \times 3\), so \(AB\) is possible and results in a \(2 \times 3\) matrix.


    \[
    AB =
    \begin{bmatrix}
    1 + 0 + 2 & -2 - 3 + 4 & 0 + 4 - 6 \\
    0 + 0 - 1 & 0 + 6 - 2 & 0 - 8 + 3
    \end{bmatrix}
    \]

    \[
    AB =
    \begin{bmatrix}
    3 & -1 & -2 \\
    -1 & 4 & -5
    \end{bmatrix}
    \]
    \item[(b)] \textbf{BC}: \(B\) is \(3 \times 3\) and \(C\) is \(1 \times 5\). Since the number of columns in \(B\) does not match the number of rows in \(C\), \(BC\) is not possible.
    \item[(c)] \textbf{AD}: \(A\) is \(2 \times 3\) and \(D\) is \(2 \times 2\). Since the number of columns in \(A\) does not match the number of rows in \(D\), \(AD\) is not possible.
\end{enumerate}

\newpage
\section*{Third part}
\subsection*{Determinant Of  \(M\)  (\(2 \times 2\))}
\[
M = \begin{bmatrix} 15 & 10 \\ 3 & 2 \end{bmatrix}
\]
\[
\det(M) = (15 \times 2) - (10 \times 3) = 30 - 30 = 0.
\]



\subsection*{Determinant Of  \(M\)  (\(3 \times 3\))}
\[
M = \begin{bmatrix} 2 & 3 & 1 \\ -1 & 2 & 3 \\ 3 & 2 & -1 \end{bmatrix}
\]
\[
\det(M) = 2 \begin{vmatrix} 2 & 3 \\ 2 & -1 \end{vmatrix} - 3 \begin{vmatrix} -1 & 3 \\ 3 & -1 \end{vmatrix} + 1 \begin{vmatrix} -1 & 2 \\ 3 & 2 \end{vmatrix}
\]
\[
\begin{vmatrix} 2 & 3 \\ 2 & -1 \end{vmatrix} = (2 \times -1) - (3 \times 2) = -2 - 6 = -8
\]
\[
\begin{vmatrix} -1 & 3 \\ 3 & -1 \end{vmatrix} = (-1 \times -1) - (3 \times 3) = 1 - 9 = -8
\]
\[
\begin{vmatrix} -1 & 2 \\ 3 & 2 \end{vmatrix} = (-1 \times 2) - (2 \times 3) = -2 - 6 = -8
\]
\[
\det(M) = (2 \times -8) - (3 \times -8) + (1 \times -8) = -16 + 24 - 8 = 0.
\]



\newpage
\section*{Fourth part}
\[
A = \begin{bmatrix} -3 & -2 \\ 3 & 3 \end{bmatrix}
\]
The inverse \(A^{-1}\) exists if \(\det(A) \neq 0\):
\[
\det(A) = (-3 \times 3) - (-2 \times 3) = -9 + 6 = -3 \neq 0
\]
The inverse is given by:
\[
A^{-1} = \frac{1}{\det(A)} \begin{bmatrix} d & -b \\ -c & a \end{bmatrix}
\]
\[
A^{-1} = \frac{1}{-3} \begin{bmatrix} 3 & 2 \\ -3 & -3 \end{bmatrix} = \begin{bmatrix} -1 & -\frac{2}{3} \\ 1 & 1 \end{bmatrix}
\]

\section*{Fifth part}
Three equations are linearly independent if:
\begin{enumerate}
    \item[(b)] There is no way to express one equation as a linear combination of the others.
    \item[(c)]   The graphical representations of the equations are lines that do not intersect. 

\end{enumerate}

\section*{Sixth part}
I am too lazy to do it, but would it not be valid if we make all the matrices \( A, B, C, D \) of size \( (1 \times 1) \)?  

\end{document}
